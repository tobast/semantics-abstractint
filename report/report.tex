\documentclass[11pt,a4paper]{article}
\usepackage[utf8]{inputenc}
\usepackage[francais]{babel} %% FRENCH
\usepackage[T1]{fontenc}
\usepackage{amsmath}
\usepackage{amsfonts}
\usepackage{amssymb}
\usepackage{graphicx}
\usepackage[left=2cm,right=2cm,top=2cm,bottom=2cm]{geometry}

% Custom packages
\usepackage{my_listings}
\usepackage{my_hyperref}
\usepackage{math}

\author{Théophile \textsc{Bastian}}
\title{Projet d'interprétation abstraite}
\date{\today}

\begin{filecontents}{reportbib.bib}
@inproceedings{cousotWiden,
  title={Comparing the Galois connection and widening/narrowing approaches to abstract interpretation},
  author={Cousot, Patrick and Cousot, Radhia},
  booktitle={Programming Language Implementation and Logic Programming},
  pages={269--295},
  year={1992},
  organization={Springer}
}
\end{filecontents}

\begin{document}
\maketitle

\begin{abstract}
	Au cours de ce projet, j'ai implémenté en OCaml un analyseur statique
	par interprétation abstraite sur les domaines des constantes et
	des intervalles. Celui-ci permet une analyse basique de fonctions
	(par sauts au point d'entrée et du point de sortie), supporte le
	\textit{narrowing} (à une seule passe, ce qui est suffisant pour
	les intervalles) et est capable de lever des alertes basiques.
\end{abstract}

\section{Fonctionnalités}
Dans tous les cas rencontrés, l'analyseur est toujours \emph{sûr}. Celui-ci
permet une analyse sur les domaines des constantes et des intervalles, supporte
l'analyse inter-procédurale ainsi que le \textit{narrowing}.

En pratique, l'opérateur de narrowing est implémenté (autrement que
trivialement) uniquement sur le domaine des intervalles. L'opérateur utilisé
est tiré de~\cite{cousotWiden} et défini par
\[
	\Delta : [a,b], [c,d] \mapsto [
		\begin{cases}c&,\text{~si } a = -\infty\\
			a&,\text{~sinon} \end{cases},
		\begin{cases}d&,\text{~si } b = +\infty\\
			b&,\text{~sinon} \end{cases}
		]
\]		

\subsection{Utilisation}

La compilation (\lstbash{make}) produira un fichier \texttt{main.native}.
Celui-ci prend plusieurs options~:
\begin{itemize}
	\item \lstbash{-constant}~: domaine des constantes~;
	\item \lstbash{-interval}~: domaine des intervalles~;
	\item \lstbash{-concrete}~: domaine concret, utilisé au début, qui n'est
		plus vraiment à jour et n'est absolument pas complet ni utilisable~;
	\item \lstbash{-dot}~: génère un fichier \texttt{.dot} du fichier analysé.
		Le graphe de flot de contrôle généré peut être affiché avec un
		\lstbash{make dot}.
\end{itemize}

Plusieurs domaines peuvent être demandés à la fois. Pour
(re)générer les fichiers \og{}\texttt{.out}~\fg{} % chktex 36
accompagnant les fichiers de test, il est possible d'utiliser
\lstc{make test}.

\subsection{Fonctionnement}

L'analyseur marche comme conseillé dans la plupart des cas. L'itérateur utilisé
fonctionne par \textit{worklist}. Toutefois, celle-ci ayant posé quelques
problèmes à cause du \textit{narrowing}/\textit{widening} en interaction
avec les valeurs initiales des domaines, j'ai été forcé de faire deux passes
d'itérations (\ie, une fois la worklist vide, on la relance à partir du point
d'entrée). Ceci ne ralentit pas significativement l'analyseur.

Je n'ai pas pu interfacer mon analyseur avec Apron, car j'ai modifié la
signature \lstocaml{DOMAIN} pour y ajouter des fonctionnalités.

\section{Analyse des exemples}
Tous les exemples se trouvent dans \texttt{examples/} et sont accompagnés
d'un fichier \texttt{.interv.out}, la sortie de l'analyse sur les intervalles,
et \texttt{.const.out}, sur le domaine des constantes.

\subsection{\texttt{bwd\_expr.c}}
Cet exemple sert à vérifier les opérateurs \og{}backward~\fg{}d'un domaine de
valeurs (\lstocaml{bwd\_unary} et \lstocaml{bwd\_binary}). Le succès de
l'assertion montre que ces opérateurs marchent (du moins pour $+$).

\subsection{\texttt{call.c}}
Cet exemple teste les appels de fonction, grâce à une fonction \lstc{add}
qui additionne et renvoie ses deux paramètres. On lui demande $2+3$ et $3+4$,
ainsi l'intervalle de retour est $[5, 7]$, d'où le fait que l'intervalle
final ne soit pas réduit à une constante.

\subsection{\texttt{call\_cfgfail.c}}
Cet exemple démontre l'existence d'une erreur dans le CFG (que je n'ai pas
cherché à réparer)~: s'il existe des variables portant le même nom, locales
dans deux fonctions différentes, le CFG ne choisit pas nécessairement la
bonne\ldots

\subsection{\texttt{constants.c}}
Ce test a été prévu pour tester le domaine des constantes
essentiellement (suite d'opérations, sans boucles ou conditions~: une variable
a toujours une valeur entière constante).

L'analyse sur les constantes produit bien le résultat voulu (comme en
témoigne l'absence d'alertes par rapport à l'assertion finale). Toutefois,
il est intéressant de noter qu'une analyse sur le domaine des intervalles,
bien que produisant le bon résultat, lève une alerte pour l'exception~:
en effet, $\neg ($\lstc{x==3}$) \equiv$ \lstc{x!=3}, ce qui donne $\top$
dans le domaine des intervalles.

\subsection{\texttt{narrow.c}}
Cet exemple teste le \textit{narrowing}~: sans celui-ci, sa sortie indiquerait
$i \in [0, +\infty]$ et $a \in [0, +\infty]$, mais l'application du
narrowing permet de se ramener aux meilleurs intervalles possibles.

\subsection{\texttt{relational.c}}
Cet exemple est à la fois négatif et positif~: d'un côté, il montre
les limites du domaine des intervalles, qui n'est pas relationnel~;
d'un autre, il montre que l'analyseur sait détecter les divisions par zéro.

\subsection{\texttt{syra.c}}
Cet exemple est un test plus général. Il calcule la suite de Syracuse,
et vérifie que si la suite passe en dessous de $2$ à un moment donné, alors
sa valeur est $1$. Cela montre également que l'analyseur ne se soucie pas de
la terminaison.

\bibliography{reportbib}{}
\bibliographystyle{plain}

\end{document}

